\documentclass[a4paper]{article}

\usepackage{xeCJK} % 使用繁體中文
\usepackage{mathtools} % 使用數學工具
\setCJKmainfont[AutoFakeSlant=.2, AutoFakeBold = 4]{AR PL KaitiM Big5} % 標楷體 (AutoFakeSlant -> 調整斜體, AutoFakeBold -> 調整粗體度)
%\setmainfont{Times New Roman}
\setCJKmonofont{AR PL KaitiM Big5}

\usepackage[left=2cm, right=2cm, top=2.0cm, bottom=1cm]{geometry}
\usepackage{xcolor, fontawesome5, graphicx, titlesec, tabularx, enumitem}
\usepackage{tikz}
\usepackage[hidelinks]{hyperref} % 支援超連結
\renewcommand{\familydefault}{\sfdefault}

\definecolor{myblue}{RGB}{40, 64, 119} % 自訂藍色
\titleformat{\section}{\Large\bfseries\color{myblue}}{}{0em}{}[\titlerule]

\pagestyle{empty} % 關閉頁碼


\begin{document}

\begin{center}
    \textbf{\large Autobiography}
\end{center}


\section*{Introduction}
I am Yao-Chih Hsu, currently pursuing a Master's degree in Statistics in the Department of Applied Mathematics at National Dong Hwa University. My research interests include neural networks, machine learning, and time series analysis. Since my undergraduate studies, I have developed a strong interest in neural networks and have actively engaged in self-directed learning through online resources. My current research focuses on time series analysis, aiming to leverage the strengths of neural networks in handling high-dimensional data and automatically extracting meaningful features to address the problem of missing values in temporal data. Furthermore, I seek to extend this work to observational locations, developing new approaches to mitigate data loss caused by machine failures or other external factors.

 
\section*{Skills}
Artificial intelligence is a key driver of future technological advancements, with its core essence being a vast mathematical model. Modern machine learning techniques not only allow deep optimization for specific domains but also offer various general-purpose models applicable across different scenarios. While general models may not always outperform specialized ones in certain fields, their predictive and analytical power can be significantly enhanced through data optimization and improved learning methods.I am proficient in data analysis and machine learning using Python and R. My research and practical experience include models such as CNN, LSTM, ARIMA, KNN, and $SSSD^{S4}$. Despite this, I strive to further enhance model generalization to ensure efficient and accurate decision support for unforeseen events.Beyond the technical domain, I have practical experience in secondary education and have conducted in-depth research on educational theories. I am well-versed in how educational theories influence instructional design and have applied this knowledge in curriculum planning. My goal is to make mathematics more comprehensible and engaging for students, transforming it from a subject often perceived as difficult into an enjoyable and accessible field of study.


\section*{Education}
During my undergraduate studies, I majored in Applied Mathematics, complementing my studies with various courses in statistics and data science to build a solid foundation in mathematics and statistics. I actively participated in academic conferences and workshops, broadening my perspective and deepening my understanding of the integration between theory and application.

In addition to my major, I pursued a secondary education program. One of my most memorable experiences was participating in educational observation and teaching internships, which allowed me to step into middle school classrooms and experience the challenges and opportunities of teaching firsthand. During my internship, I experimented with integrating mathematical concepts with multimedia, using interactive presentations and real-life examples to guide students in learning. I closely observed students' learning behaviors and adjusted my teaching strategies accordingly. I firmly believe that the combination of theoretical knowledge and practical experience enriches students' learning journeys.

Additionally, I have established a personal website to document and share my learning experiences in data science and machine learning. Through this platform, I aim to track my learning progress, showcase my practical projects, and continually deepen my expertise.



\section*{Work Experience}
Education is a long-term endeavor and a process of intellectual enlightenment. I aspire to transform complex knowledge into simple and engaging lessons, helping students realize that "science is present in all aspects of life," thereby inspiring their motivation to learn. Mathematics has often been perceived as a difficult subject, but by starting from fundamental concepts and gradually linking mathematical theories to real-world applications, it can be transformed from a collection of abstract formulas into a key that unlocks understanding and a powerful tool for problem-solving.

Currently, I serve as a part-time mathematics teacher at a junior high school. In my teaching, I guide students through questioning and dialogue, helping them understand problems, reflect on them, and develop logical reasoning skills. By incorporating real-life examples and hands-on exercises, I assist students in translating real-world issues into classroom problems. I firmly believe that mathematics and data science should not be confined to abstract theories but should serve as essential tools for addressing practical challenges.



\section*{Future Outlooks}
Looking ahead, I aspire to build bridges among academia, industry, and education by applying mathematics and data science to real-world problems. I am committed to promoting the accessibility and adoption of data science education, enabling more people to understand and effectively utilize these technologies. Ultimately, I hope to transform knowledge into a driving force for societal and technological advancement, contributing to a more efficient and intelligent way of life.

The 108 Curriculum Guidelines emphasize "lifelong learning," highlighting not only the importance of solving real-world problems but also the ability to adapt to future challenges. The current educational reform focuses on "diverse development and problem-solving," aligning with my philosophy of learning. According to the Ministry of Education, core competencies encompass the "knowledge," "skills," and "attitudes" necessary to navigate present and future challenges. These competencies extend beyond academic knowledge and skills, emphasizing the integration of learning with real life. Through practical application, learners can achieve holistic development. I firmly believe that for both myself and my students, discovering personal strengths, identifying career interests, and cultivating self-directed learning abilities are of utmost importance.

I aspire to further expand my interdisciplinary learning, exploring the world from diverse perspectives and applying my knowledge to real-world scenarios, ultimately enriching both my own and my students' learning experiences.



\end{document}