\documentclass[a4paper]{article}

\usepackage{xeCJK} % 使用繁體中文
\usepackage{mathtools} % 使用數學工具
\setCJKmainfont[AutoFakeSlant=.2, AutoFakeBold = 4]{AR PL KaitiM Big5} % 標楷體 (AutoFakeSlant -> 調整斜體, AutoFakeBold -> 調整粗體度)
%\setmainfont{Times New Roman}
\setCJKmonofont{AR PL KaitiM Big5}

\usepackage[left=2cm, right=2cm, top=2.0cm, bottom=1cm]{geometry}
\usepackage{xcolor, fontawesome5, graphicx, titlesec, tabularx, enumitem}
\usepackage{tikz}
\usepackage[hidelinks]{hyperref} % 支援超連結
\renewcommand{\familydefault}{\sfdefault}

\definecolor{myblue}{RGB}{40, 64, 119} % 自訂藍色
\titleformat{\section}{\Large\bfseries\color{myblue}}{}{0em}{}[\titlerule]

\pagestyle{empty} % 關閉頁碼


\begin{document}

\begin{center}
    \textbf{\Large 自傳}
\end{center}


\section*{前言}
我是許堯智,現為國立東華大學應用數學系統計碩士,研究方向為神經網路、機器學習與時間序列。自大學時期起,我對神經網路產生濃厚興趣,並自行上網尋找相關資料學習。目前,我的研究方向聚焦於時間序列,希望藉由神經網路處理高維度資料、自動找出特徵值等優勢,研究填補時間資料缺值的方法,並拓展至為觀測地點,為因機器缺失或其他因素所導致的資料缺失找到新的解方。

 
\section*{專業技能}
人工智慧是未來科技發展的關鍵,而其核心本質則是一套龐大的數學模型。當今的機器學習技術,除了能針對特定領域進行深度優化,還擁有許多通用模型,能夠靈活應用於不同的場景。雖然通用模型在某些專業領域不一定能超越專用模型,但透過資料與學習方式的優化,仍能發揮出強大的預測與分析能力。我擅長使用 Python 和 R 進行資料分析與機器學習,並具備 CNN、LSTM、ARIMA、KNN、$SSSD^{S4}$ 等模型的研究與實作經驗。即使如此,我仍希望能進一步提升模型的泛化能力,使其在預測未知事件時可以提供高效且準確的決策支援。除了技術領域,我亦擁有中等教育的實踐經驗,並深入研究教育相關理論。我熟習教育理論對教學設計的影響,並將其應用於學校課程的規劃,期望透過符合學生認知發展的方式,以簡單的範例,讓學生不再排斥數學,甚至能培養興趣,使數學知識更加易懂、更具吸引力。


\section*{求學過程}
在大學期間,我主修應用數學,並選修多門統計與資料科學相關課程,為後續的研究奠定扎實的數學與統計基礎。同時,我積極參與校內外研討會與工作坊,如南區統計研討會等,開拓視野、深化理論與應用的結合。

除了本科系的學習,我亦修習中等教育學程。印象最深的是參與教育見習與教育實習,讓我有機會實際走入中學教室授課,體驗教學現場的挑戰與機遇。在教育實習期間,我嘗試將數學概念結合多媒體,運用互動式簡報與生活化的題材引導學生學習,並觀察學生的學習行為,以適時調整教學策略。我相信理論與現場經驗的融合,能夠帶給學生更豐富的學習過程。

此外,我也著手建立個人網站,分享資料科學、機器學習等所學與實作經驗,期望透過網路平台記錄自己的學習軌跡、展示實作成果,並促使自己持續深耕專業,不斷成長。



\section*{職涯經歷}
教育是一項百年大計,更是啟發智慧的過程。我希望能將所學轉化為簡單且有趣的課程,讓學生認識到「原來生活中處處充滿科學」,進而激發學習的動力。數學或許曾是許多學生的夢魘,但當我們從基本概念出發,逐步將數學理論與實際應用結合時,數學便不再只是枯燥的公式,而是一把開啟世界大門的鑰匙,也是一個用於解決任何問題的工具。

目前,我擔任國民中學數學兼課教師。在教學中,我善於透過提問與對話引導學生認識、理解到反思題目,藉由培養邏輯推理能力,並透過實例與實作,幫助學生將生活中的問題轉化為課堂上的習題。我相信,數學與資料科學不應只是抽象的理論,而應成為解決現實問題的重要工具。



\section*{未來展望}
我希望能在學術界、產業界與教育界之間搭建橋樑,將數學與資料科學應用於實際問題,並持續推動資料科學教育的普及,讓更多人理解並善用這項技術;同時將知識轉化為推動文明進步的力量,實現更加便捷與智慧的生活。

108 課綱強調「做一個終身學習者」,不僅要解決生活問題,更要具備迎接未來挑戰的能力。「多元發展,解決問題」是當前教育改革的核心方向,這也與我的學習理念不謀而合。教育部指出,核心素養是指一個人,為了適應現在生活及未來挑戰,所應具備的「知識」、「能力」與「態度」,核心素養不僅關乎學科知識與技能,更強調學習與生活的結合,透過實踐來展現學習者的全人發展。我深信,無論是對於我自己或我的學生而言,發掘自身特長與職業興趣,以及培養自主學習的能力,都是至關重要的。

拓展跨領域學習,透過不同視角來探索世界,並將所學應用於實務,帶給自己與學生更多元化的人生歷程。



\end{document}