\documentclass[a4paper]{article}

\usepackage{xeCJK} % 使用繁體中文
\usepackage{mathtools} % 使用數學工具
\setCJKmainfont[AutoFakeSlant=.2, AutoFakeBold = 4]{AR PL KaitiM Big5} % 標楷體 (AutoFakeSlant -> 調整斜體, AutoFakeBold -> 調整粗體度)
%\setmainfont{Times New Roman}
\setCJKmonofont{AR PL KaitiM Big5}

\usepackage[left=2cm, right=2cm, top=0.0cm, bottom=1cm]{geometry}
\usepackage{xcolor, fontawesome5, graphicx, titlesec, tabularx, enumitem}
\usepackage{tikz}
\usepackage[hidelinks]{hyperref} % 支援超連結
\renewcommand{\familydefault}{\sfdefault}

\definecolor{myblue}{RGB}{40, 64, 119} % 自訂藍色
\titleformat{\section}{\Large\bfseries\color{myblue}}{}{0em}{}[\titlerule]

\pagestyle{empty} % 關閉頁碼


\begin{document}

% Header
\begin{minipage}{0.25\textwidth}
    \vspace*{-\topskip} % 去除最上方的空白
    \begin{tikzpicture}
        \clip (0,0.8) circle(2.5cm); % 圓形裁剪區域
        \node[anchor=center] at (0,0) {\includegraphics[width=2.5cm]{../../img/head.jpg}}; % 個人大頭照 (請更換)
    \end{tikzpicture}
\end{minipage}
\hfill
\begin{minipage}{0.65\textwidth}
    \raggedright
    \vspace{1.3cm}
    \hspace{2cm}
    {\fontsize{35}{48}\selectfont \bfseries \color{myblue} 許 \ 堯 \ 智}\\[1em] % 使名字更大
    \vspace{0.1cm} % 讓名字和電話之間有一點間距

    \begin{minipage}{0.6\textwidth}
        \faGlobe\ \href{https://yao-chih.netlify.app}{https://yao-chih.netlify.app} \quad \\
        \faLinkedin\ \href{https://www.linkedin.com/in/yao-chih}{https://www.linkedin.com/in/yao-chih} \\
        \faGithub\ \href{https://github.com/Josh-test-lab}{https://github.com/Josh-test-lab} \quad \\
        \faEnvelope\ \href{mailto:hyc0113@hlc.edu.tw}{hyc0113@hlc.edu.tw}
    \end{minipage}
    \hfill
    \begin{minipage}{0.3\textwidth}
        \centering
        \includegraphics[width=2cm]{../../img/website QRcode.png}
    \end{minipage}
    
\end{minipage}



\section*{ABOUT ME}

~~~~~積極學習,熱愛新事物,期望豐富人生經驗。現為國立東華大學應用數學系統計碩士,研究神經網路、機器學習與時間序列。  

碩士研究方向為時間序列與空間插補,利用神經網路處理高維度資料,透過特徵提取進行未來預測;同時經由座標關係,為缺失地點進行插值,以彌補因機器缺失或其他因素所導致的資料缺失。

教育是啟發智慧的關鍵。我希望將所學轉化為簡單有趣的課程,讓學生發現「科學無處不在」,激發學習動力。數學或許讓人卻步,但從基本概念出發,逐步結合理論與實務,那數學便不再只是枯燥的公式,而是一把打開世界大門的鑰匙。

\section*{EXPERIENCE}
\begin{minipage}[t]{0.25\textwidth}
    \begin{itemize}[left=0pt]
        \item \textbf{2025.02 - 現在}
    \end{itemize}
\end{minipage}
\hfill
\begin{minipage}[t]{0.3\textwidth}
    \textbf{花蓮縣立壽豐國民中學}
\end{minipage} 
\hfill
\begin{minipage}[t]{0.4\textwidth}
    \begin{itemize}[left=1em, itemsep=0pt, parsep=0pt]
        \item 數學兼課教師
    \end{itemize}
\end{minipage} 
\hfill
\vspace{0.5cm}
\begin{minipage}[t]{0.25\textwidth}
    \begin{itemize}[left=0pt]
        \item \textbf{2024.02 - 現在}
    \end{itemize}
\end{minipage}
\hfill
\begin{minipage}[t]{0.3\textwidth}
    \textbf{國立東華大學}
\end{minipage} 
\hfill
\begin{minipage}[t]{0.4\textwidth}
    \begin{itemize}[left=1em, itemsep=0pt, parsep=0pt]
        \item 研究助理
        \item 時間序列與空間研究
        \item 缺失觀測資料填補研究
    \end{itemize}
\end{minipage} 
\hfill
\vspace{0.5cm}
\begin{minipage}[t]{0.25\textwidth}
    \begin{itemize}[left=0pt]
        \item \textbf{2023.09 - 2025.01}
    \end{itemize}
\end{minipage}
\hfill
\begin{minipage}[t]{0.3\textwidth}
    \textbf{花蓮縣立壽豐國民中學}
\end{minipage} 
\hfill
\begin{minipage}[t]{0.4\textwidth}
    \begin{itemize}[left=1em, itemsep=0pt, parsep=0pt]
        \item 學習扶助、縣課輔、假日陪讀、教學
    \end{itemize}
\end{minipage} 
\hfill
\vspace{0.5cm}
\begin{minipage}[t]{0.25\textwidth}
    \begin{itemize}[left=0pt]
        \item \textbf{2023.08 - 2024.01}
    \end{itemize}
\end{minipage}
\hfill
\begin{minipage}[t]{0.3\textwidth}
    \textbf{中等學校教育實習}
\end{minipage} 
\hfill
\begin{minipage}[t]{0.4\textwidth}
    \begin{itemize}[left=1em, itemsep=0pt, parsep=0pt]
        \item 將邏輯推斷、數理知識融入課堂
        \item 行政校務工作
        \item 引導、帶領學生參加小論文競賽、發明展競賽、資訊教育競賽等
    \end{itemize}
\end{minipage} 
\hfill
\vspace{-0.5cm}

\section*{EDUCATION}
\begin{itemize}[left=0pt]
    \item \textbf{2024 - 現在} \hfill 應用數學系 \ 統計碩士班 \ 碩士在學 \hfill 國立東華大學

    \item \textbf{2021 - 2023} \hfill 教育學程 \ 修畢 \hfill 國立東華大學

    \item \textbf{2019 - 2023} \hfill 應用數學系 \ 統計科學組 \ 學士畢業 \hfill 國立東華大學
\end{itemize}

\section*{PROJECT}
\begin{itemize}[left=0pt]
    \item \textbf{R 套件 `autoFRK` Python 版本改寫}
    \texttt{https://pypi.org/project/autoFRK/}
    \item \textbf{花蓮縣 113 年度資訊教育競賽活動貓咪盃 Scratch 遊戲/動畫遊戲程式設計競賽}(指導教師)
    \item \textbf{臺鐵志學站停車場停車模擬模型專題}(碩士課程)\\
    \texttt{https://yao-chih.netlify.app/parking-model-for-the-parking-lot-in-zhixue-station}
    \item \textbf{紅樓夢人物分析專題}(學士專題)\\
    \texttt{https://yao-chih.netlify.app/the-analysis-of-characters-in-dream-of-the-red-chamber}
\end{itemize}


\section*{\begin{minipage}{0.40\textwidth} INDIVIDUAL TRAITS \end{minipage}\hfill \begin{minipage}{0.34\textwidth} EXPERTISE \end{minipage}\hfill \begin{minipage}{0.25\textwidth} KEEP LEARNING \end{minipage}}
\begin{minipage}{0.38\textwidth}
\begin{itemize}[left=0pt]
    \item 細微觀察
    \item 邏輯推理
    \item 富責任感
    \item 團隊合作
\end{itemize}
\end{minipage}
\hfill
\begin{minipage}{0.32\textwidth}
\begin{itemize}[left=0pt]
    \item 數學邏輯
    \item 資料處理
    \item 程式設計(Python \& R)
    \item 學生教育
\end{itemize}
\end{minipage}
\hfill
\begin{minipage}{0.25\textwidth}
\begin{itemize}[left=0pt]
    \item 機器學習
    \item 時間序列
    \item 空間模型
    \item 缺值填補
\end{itemize}
\end{minipage}

\end{document}